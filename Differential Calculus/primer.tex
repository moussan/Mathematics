\documentclass{article}
\usepackage{graphicx} % Required for inserting images
\usepackage{amsmath}    % Required for maths

\title{Differential Calculus Primer}
\author{Moussa El Najmi}
\date{February 2025}

\begin{document}

\maketitle

\section{Introduction}
\paragraph{Here’s an advanced primer on \textbf{Differential Calculus}, covering rules, equations, formulas, and techniques comprehensively.}
\newpage
\section{Foundations of Differential Calculus}
\subsection{Definition of a Derivative}
For a function \( f(x) \), the \textbf{derivative} is defined as:
\[
f'(x) = \lim_{h \to 0} \frac{f(x+h) - f(x)}{h}
\]
This gives the \textbf{instantaneous rate of change} of \( f(x) \) at \( x \).
\subsection{Alternative notations:}
\[
\frac{d}{dx} f(x), \quad Df(x), \quad \dot{f}(x) \text{ (Newton's notation for time derivatives)}
\]
\newpage
\section{Basic Differentiation Rules}
\subsection{Power Rule}
\[
\frac{d}{dx} x^n = n x^{n-1}, \quad \text{for } n \in \mathbb {R}
\]
\subsection{Constant Rule}
\[
\frac{d}{dx} C = 0, \quad C \text{ is a constant}
\]
\subsection{Constant Multiple Rule}
\[
\frac{d}{dx} [c f(x)] = c f'(x), \quad c \text{ is a constant}
\]
\subsection{Sum and Difference Rule}
\[
\frac{d}{dx} [f(x) \pm g(x)] = f'(x) \pm g'(x)
\]
\subsection{Product Rule}
\[
\frac{d}{dx} [f(x) g(x)] = f'(x) g(x) + f(x) g'(x)
\]
\subsection{Quotient Rule}
\[
\frac{d}{dx} \left[\frac{f(x)}{g(x)}\right] = \frac{f'(x)g(x) - f(x)g'(x)}{[g(x)]^2}, \quad g(x) \neq 0
\]
\subsection{Chain Rule (Composite Functions)}
If \( y = f(g(x)) \), then:
\[
\frac{dy}{dx} = f'(g(x)) g'(x)
\]
\newpage
\section{Advanced Differentiation Techniques}

\subsection{Implicit Differentiation}

Used when \( y \) is given implicitly in terms of \( x \):  
\[
\frac{d}{dx} [F(x, y) = 0] \Rightarrow F_x + F_y \frac{dy}{dx} = 0
\]

Example: Differentiate \( x^2 + y^2 = 1 \)  
\[
2x + 2y \frac{dy}{dx} = 0 \Rightarrow \frac{dy}{dx} = -\frac{x}{y}
\]

\subsection{Logarithmic Differentiation}

Used for functions of the form \( f(x)^{g(x)} \) or complicated products:

1. Take natural log:  
   \[
   y = f(x)^{g(x)} \Rightarrow \ln y = g(x) \ln f(x)
   \]

2. Differentiate both sides using the product rule.

3. Solve for \( \frac{dy}{dx} \).

Example:
\[
y = x^x \Rightarrow \ln y = x \ln x
\]
Differentiate:
\[
\frac{1}{y} \frac{dy}{dx} = \ln x + 1
\]
Multiply by \( y \):
\[
\frac{dy}{dx} = x^x (\ln x + 1)
\]

\newpage
\section{Higher-Order Derivatives}
 
\subsection{Second Derivative}
\[
f''(x) = \frac{d^2 y}{dx^2} = \frac{d}{dx} \left( \frac{dy}{dx} \right)
\]

\subsection{Nth Order Derivative}
\[
f^{(n)}(x) = \frac{d^n y}{dx^n}
\]

Example:  
If \( f(x) = e^x \), then:
\[
f^{(n)}(x) = e^x
\]

\newpage
\section{Differentiation of Special Functions}

\subsection{Trigonometric Functions}
\[
\frac{d}{dx} \sin x = \cos x, \quad \frac{d}{dx} \cos x = -\sin x
\]
\[
\frac{d}{dx} \tan x = \sec^2 x, \quad \frac{d}{dx} \cot x = -\csc^2 x
\]

\subsection{Inverse Trigonometric Functions}
\[
\frac{d}{dx} \sin^{-1} x = \frac{1}{\sqrt{1-x^2}}, \quad \frac{d}{dx} \cos^{-1} x = -\frac{1}{\sqrt{1-x^2}}
\]
\[
\frac{d}{dx} \tan^{-1} x = \frac{1}{1+x^2}, \quad \frac{d}{dx} \cot^{-1} x = -\frac{1}{1+x^2}
\]

\subsection{Exponential and Logarithmic Functions}
\[
\frac{d}{dx} e^x = e^x, \quad \frac{d}{dx} a^x = a^x \ln a
\]
\[
\frac{d}{dx} \ln x = \frac{1}{x}, \quad \frac{d}{dx} \log_a x = \frac{1}{x \ln a}
\]

\newpage
\section{Applications of Differential Calculus}
\subsection{Critical Points \& Extrema}
Find local maxima/minima using:
- First Derivative Test: \( f'(x) = 0 \) and sign analysis.
- Second Derivative Test:  
  - \( f''(x) > 0 \) → Local minimum
  - \( f''(x) < 0 \) → Local maximum

\subsection{Concavity \& Inflection Points}
\[
\text{If } f''(x) > 0, \text{ function is concave up (convex)}
\]
\[
\text{If } f''(x) < 0, \text{ function is concave down (concave)}
\]

\subsection{L'Hôpital’s Rule (Indeterminate Forms)}

If \( \lim_{x \to a} \frac{f(x)}{g(x)} \) results in 0/0 or \( \infty/\infty \),  
then:
\[
\lim_{x \to a} \frac{f(x)}{g(x)} = \lim_{x \to a} \frac{f'(x)}{g'(x)}
\]

\subsubsection{Taylor and Maclaurin Series}

Approximates a function using derivatives at a point \( a \):
\[
f(x) = f(a) + f'(a)(x-a) + \frac{f''(a)}{2!} (x-a)^2 + \cdots
\]

Maclaurin Series (at \( a = 0 \)):
\[
f(x) = \sum_{n=0}^{\infty} \frac{f^{(n)}(0)}{n!} x^n
\]

Example:  
\[
e^x = \sum_{n=0}^{\infty} \frac{x^n}{n!}
\]

\newpage
\section{Advanced Topics}

\subsection{Differentiation under the Integral Sign (Leibniz Rule)}

If \( I(a) = \int_{a}^{b} f(x, a) dx \), then:
\[
\frac{d}{da} I(a) = \int_{a}^{b} \frac{\partial}{\partial a} f(x, a) dx
\]

\subsection{Functional Derivatives  }

For functionals \( F[y] \), the Euler-Lagrange equation gives extremal functions:
\[
\frac{\delta F}{\delta y} = 0
\]
\newpage
\paragraph{This covers core rules, techniques, and applications of Differential Calculus at an advanced level.}
\end{document}
