\documentclass{article}
\usepackage{graphicx} % Required for inserting images
\usepackage{amsmath}    % Required for maths

\title{Integral Calculus Primer}
\author{Moussa El Najmi}
\date{February 2025}

\begin{document}

\maketitle
\section{Introduction}
\paragraph{Here’s an advanced primer on Integral Calculus, covering rules, equations, formulas, and techniques in detail.}
\newpage
\section{Fundamentals of Integral Calculus}
\subsection{Definition of an Integral}
\subsubsection{Indefinite Integral (Antiderivative)}
The indefinite integral of \( f(x) \) is a function \( F(x) \) such that:
\[
\int f(x) \, dx = F(x) + C
\]
where \( C \) is the constant of integration.
\subsubsection{Definite Integral}
The definite integral of \( f(x) \) over \( [a, b] \) is:
\[
\int_a^b f(x) \, dx = F(b) - F(a)
\]
where \( F(x) \) is an antiderivative of \( f(x) \).
\newpage
\section{Fundamental Theorem of Calculus}
\subsection{Differentiation of an Integral:}
   \[
   \frac{d}{dx} \int_a^x f(t) dt = f(x)
   \]
\subsection{Evaluation of a Definite Integral:}
   \[
   \int_a^b f(x) dx = F(b) - F(a)
   \]
\newpage
\section{Basic Integration Rules}
\subsection{Power Rule}
\[
\int x^n dx = \frac{x^{n+1}}{n+1} + C, \quad n \neq -1
\]
\subsection{Constant Rule}
\[
\int c \, dx = cx + C
\]
\subsection{Sum/Difference Rule}
\[
\int [f(x) \pm g(x)] dx = \int f(x) dx \pm \int g(x) dx
\]
\subsection{Constant Multiplication Rule}
\[
\int c f(x) dx = c \int f(x) dx
\]
\subsection{Integration by Substitution (Change of Variables)}
If \( u = g(x) \), then:
\[
\int f(g(x)) g'(x) dx = \int f(u) du
\]

Example:
\[
\int x e^{x^2} dx
\]
Let \( u = x^2 \), so \( du = 2x dx \).
\[
\frac{1}{2} \int e^u du = \frac{1}{2} e^u + C = \frac{1}{2} e^{x^2} + C
\]
\newpage
\section{Advanced Integration Techniques}
\subsection{Integration by Parts}
\[
\int u \, dv = uv - \int v \, du
\]
Choose:
- \( u \) = a function that simplifies when differentiated.
- \( dv \) = a function that is easy to integrate.

Example:
\[
\int x e^x dx
\]
Let \( u = x \), \( dv = e^x dx \), then:
\[
du = dx, \quad v = e^x
\]
\[
\int x e^x dx = x e^x - \int e^x dx = x e^x - e^x + C
\]

\subsection{Trigonometric Integrals}
Integrate expressions involving sin, cos, tan, sec, csc, cot using identities.

Example:
\[
\int \sin^3 x \cos x dx
\]
Use \( u = \sin x \), so \( du = \cos x dx \):
\[
\int u^3 du = \frac{u^4}{4} + C = \frac{\sin^4 x}{4} + C
\]

\subsection{Trigonometric Substitution}
Used for integrals involving:
- \( \sqrt{a^2 - x^2} \Rightarrow x = a \sin \theta \)
- \( \sqrt{a^2 + x^2} \Rightarrow x = a \tan \theta \)
- \( \sqrt{x^2 - a^2} \Rightarrow x = a \sec \theta \)

Example:
\[
\int \frac{dx}{\sqrt{9 - x^2}}
\]
Let \( x = 3 \sin \theta \), then \( dx = 3 \cos \theta d\theta \).
\[
\int \frac{3 \cos \theta d\theta}{\sqrt{9 - 9\sin^2 \theta}}
\]
Since \( \sqrt{9(1 - \sin^2 \theta)} = 3 \cos \theta \), it simplifies to:
\[
\int d\theta = \theta + C = \sin^{-1} \left(\frac{x}{3} \right) + C
\]

\subsection{Partial Fraction Decomposition}
Used when integrating rational functions:
\[
\int \frac{P(x)}{Q(x)} dx
\]
where \( Q(x) \) is factored.

Example:
\[
\int \frac{3x+5}{(x+1)(x+2)} dx
\]
Decompose:
\[
\frac{3x+5}{(x+1)(x+2)} = \frac{A}{x+1} + \frac{B}{x+2}
\]
Solving for \( A, B \), we integrate separately.

\subsection{Improper Integrals}
If \( \int_a^\infty f(x) dx \) or \( \int_{-\infty}^{\infty} f(x) dx \), take limits.

Example:
\[
\int_1^\infty \frac{1}{x^2} dx
\]
\[
\lim_{b \to \infty} \left[ -\frac{1}{x} \right]_1^b
\]
\[
\lim_{b \to \infty} \left( -\frac{1}{b} + 1 \right) = 1
\]
\newpage
\section{Special Integrals}
\subsection{Gaussian Integral}
\[
\int_{-\infty}^{\infty} e^{-x^2} dx = \sqrt{\pi}
\]
\subsection{Beta Function}
\[
B(x,y) = \int_0^1 t^{x-1} (1-t)^{y-1} dt
\]
\subsection{Gamma Function}
\[
\Gamma(n) = \int_0^\infty x^{n-1} e^{-x} dx, \quad \Gamma(n) = (n-1)!
\]
\subsection{Dirichlet Integrals}
\[
\int_0^\infty \frac{\sin x}{x} dx = \frac{\pi}{2}
\]
\newpage
\section{Applications of Integration}
\subsection{Area Under a Curve}
\[
A = \int_a^b f(x) dx
\]
\subsection{Arc Length}
\[
L = \int_a^b \sqrt{1 + \left(\frac{dy}{dx}\right)^2} dx
\]
\subsection{Surface Area of Revolution}
\[
S = \int_a^b 2\pi f(x) \sqrt{1 + \left(\frac{dy}{dx}\right)^2} dx
\]
\subsection{Volume by Integration}
- Disk Method:
  \[
  V = \pi \int_a^b [f(x)]^2 dx
  \]
- Shell Method:
  \[
  V = 2\pi \int_a^b x f(x) dx
  \]
\newpage
\paragraph{This primer covers the rules, techniques, and advanced applications of Integral Calculus.}
\end{document}
